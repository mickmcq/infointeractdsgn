\usepackage{beamerarticle}
\setjobnamebeamerversion{main-slides}
\setbeamertemplate{frametitle}
{
  \begin{centering}
    \color[rgb]{0.0,0.2,0.3}
    \textit{\insertframetitle}
    \par
  \end{centering}
}
\usepackage[left=6.75in,right=1.00in,top=0.5in,bottom=0.75in,paperheight=8.5in,paperwidth=11.5in]{geometry}
\usepackage{comment}
% Uncomment following to process comment environments
% \includecomment{comment}

% begin original titlesec segment
\begin{comment}
\usepackage{titlesec}
%\titleformat{cmnd    }[shape]{format             }{label}{sep}{before}[after]
\definecolor{sectcolor}{rgb}{0.0,0.3,0.4}
\def\thesection{\Roman{section}}
%\def\thesubsection{\arabic{subsection}}
\def\thesubsection{(\alph{subsection})}
\titleformat{\section}[block]
  {\scshape\fillast}
  {\bfseries\footnotesize\textcolor{sectcolor}\thesection\enspace}
  {1ex minus 0.1ex}
  {\color{sectcolor}\normalsize}
\titlespacing{\section}{3pc}{*3}{*2}[3pc]
\titleformat{\subsection}[hang]
  {}
  {\small\bfseries\thesubsection\enspace}
  %{1ex minus 0.1ex}
  {0ex}
  %{\small}
  {}
\titlespacing{\subsection}{0pc}{*1}{*0}[3pc]
\usepackage{titletoc}
\contentsmargin{60pt}
\titlecontents*{section}[1.5pc]
{\small}{\scshape\thecontentslabel\ }{\scshape}
{,~\thecontentspage}[.\quad ][. ]

\titlecontents*{subsection}[1.5pc]
{\scriptsize}{\itshape}{\itshape}
{,~\thecontentspage}[.\quad ][.\quad ][]
% above is based on p. 19 of titlesec.pdf
\end{comment}
% end original titlesec segment



%%%%%%%%%%%%%%%%%%%%%%%%%%%%%%%%%%%%%%%%%%%%%%%%%%%%%%%%%%%%%%%%%%%%%%

% titlesec defines the appearance of section / subsection headings {{{
\usepackage{titlesec}
\newcommand{\periodafter}[1]{#1.}
\def\thesection {\Roman{section}.}
\def\thesubsection {\arabic{subsection}.}
\def\thesubsubsection {\arabic{subsubsection}.}
%\titleformat{command}[shape]{format}{label}{sep}{before}[after]
\titleformat{\section}[block]
  {\scshape\fillast}
  {\bfseries\footnotesize\textcolor{sectcolor}\thesection\enspace}
  {1ex minus 0.1ex}
  {\color{sectcolor}\normalsize}

%\titleformat{\section}{\normalfont\bf}{\makebox[2em][r]{\thesection}}{0.5em}{\periodafter}
\titleformat{\subsection}[runin]{\normalfont\bf}{\makebox[2em][r]{\thesection\thesubsection}}{0.5em}{\periodafter}
%\titleformat{\subsubsection}[runin]{\normalfont\bf}{\makebox[2em][r]{\thesection\thesubsection\thesubsubsection}}{0.5em}{\periodafter}
\titleformat{\subsubsection}[runin]{\normalfont\bf}{\makebox[2em][r]{}}{0.5em}{\periodafter}
%  format    \titlespacing*{<command>}{<left>}{<beforesep>}{<aftersep>}[<right>]
%  star      kills indentation of subsequent paragraph
%  command   is \section, \subsection, etc.
%  left      is left margin except in the ...margin and drop shape
%  beforesep is vertical space before title
%  aftersep  is separation between title and text
%  right     is right margin (only for hang, block, or display shape)
%\titlespacing*{\section}{-2.5em}{1.3ex plus 1ex minus .2ex}{1.3ex plus .2ex}
\titlespacing*{\section}{1.5em}{1.3ex plus 1ex minus .2ex}{1.3ex plus .2ex}
\titlespacing{\subsection}{1.5em}{0em}{1.3ex plus .2ex}
\titlespacing{\subsubsection}{1.5em}{0em}{1.3ex plus .2ex}
%}}} end titlesec
% begin titletoc {{{
\usepackage{titletoc}
\contentsmargin{0pt}

\titlecontents{section}[50pt]% left
{\addvspace{2pt}}% above
{\small\scshape\llap{\thecontentslabel}\ % before with label
\small}
{\small\scshape}% before without label
{\titlerule*[1pc]{.}\footnotesize~\thecontentspage.} % filler and page
[\addvspace{1pt}]% after

%\titlecontents{section}[-2pt]% left
%{\addvspace{3pt}}% above
%{\small\scshape\llap{\thecontentslabel}\ % before with label
%\small}
%{\small\scshape}% before without label
%{,\footnotesize~\thecontentspage.} % filler and page
%[\addvspace{2pt}]% after


\titlecontents*{subsection}[2pt]
{\small\filright}{\scshape\thecontentslabel\ }{\scshape}
{,\footnotesize~\thecontentspage}[.\quad ][. ]

\titlecontents*{subsubsection}[1.5pc]
{\footnotesize\filright}{\scshape(\thecontentslabel)\ }{\scshape}
{,~\thecontentspage}[.\quad ][.\quad ][]
% Following is a kluge because I've elsewhere defined
% that a dot follows a section name. If I make
% contentsname empty, a dot will appear by itself. If I
% avoid defining contentsname, the LaTeX default
% definition takes over. I can't seem to gobble or
% phantom, but making it all white causes it to
% disappear, although it now consumes some vertical
% space.
\renewcommand\contentsname{\color{white}}
\setcounter{tocdepth}{2}
% end titletoc }}}

\usepackage[nottoc]{tocbibind}
%%%%%%%%%%%%%%%%%%%%%%%%%%%%%%%%%%%%%%%%%%%%%%%%%%%%%%%%%%%%%%%%%%



\usepackage[bookmarks=true]{hyperref}
\hypersetup{colorlinks,citecolor=black,linkcolor=blue!50!darkgray}
%\renewcommand\contentsname{}

\usepackage[english]{babel}
\usepackage{rotating}
\usepackage{hanging}
%
% BEGIN FONTS
%
% Most of what follows came from googling xelatex math
% fonts and Minion Pro and trying a lot of things that
% did not work including the minionpro package,
% mathspec, mathdesign, and more
%
% The main thing that does seem to work is to explicitly
% identify the location of the otf fonts: ~/Library/Fonts/
%
\usepackage[T1]{fontenc}
%\usepackage{amsmath}
%\usepackage{amssymb}
% Following redefines \question to produce a question
% mark but does not execute the redefinition until
% *after* the preamble---hence, \question must be
% defined in the document body
\usepackage[math-style=ISO]{unicode-math} % try sans-style=upright
\usepackage{scalefnt}
\usepackage{xltxtra}
\defaultfontfeatures{Scale=MatchLowercase,Mapping=tex-text}
% MinionPro as main font
\setmainfont[
  Path = /Users/mjmics/Library/Fonts/,
  BoldFont = MinionPro-Bold.otf,
  ItalicFont = MinionPro-It.otf,
  BoldItalicFont = MinionPro-BoldIt.otf
]{MinionPro-Regular.otf}
% MyriadPro as sans font
\setsansfont[
  Numbers={OldStyle,Proportional},LetterSpace=3,
  Path = /Users/mjmics/Library/Fonts/
]{MyriadPro-Regular.otf}
% Source Code Pro as monofont
\setmonofont{Source Code Pro}
% Tex Gyre Termes as math font except ...
\setmathfont{texgyretermes-math.otf}
% Use as much of MinionPro and MyriadPro in math as possible
\setmathfont[
  range=\mathit/{num,latin,Latin,greek,Greek},
  Path = /Users/mjmics/Library/Fonts/
]{MinionPro-It.otf}
%\setmathfont[
%  range=\mathsf/{num,latin,Latin,greek,Greek},
%  Path = /Users/mjmics/Library/Fonts/
%]{MyriadPro-Regular.otf}
\setmathfont[
  range=\mathsfit/{num,latin,Latin,greek,Greek},
  Path = /Users/mjmics/Library/Fonts/
]{MyriadPro-It.otf}
\setmathfont[
  range=\mathbfsfit/{num,latin,Latin,greek,Greek},
  Path = /Users/mjmics/Library/Fonts/
]{MyriadPro-BoldIt.otf}
\let\mathbf\mathbfsf
\let\mathbfit\mathbfsfit
\usepackage{pifont}
% END FONTS
%
%\everymath{\fam 1\relax}
%\everydisplay{\fam 1\relax}
\def\log{\mathop{\rm log}\nolimits}
\usepackage[font=sc,labelsep=period,format=hang]{caption}
\usepackage[ragged]{sidecap}
\definecolor{title}{RGB}{20,180,20}
\definecolor{sectcolor}{rgb}{0.0,0.3,0.4}
\definecolor{solutioncolor}{rgb}{0.3,0.5,0.4}
\definecolor{solutionNcolor}{rgb}{0.1,0.4,0.1}
\definecolor{authordatecolor}{RGB}{150,150,150}
\definecolor{remarkcolor}{rgb}{0.5,0.5,0.5}
\definecolor{OrangeRed}{rgb}{1,0.27,0}
\newcommand{\ored}[1]{\textcolor{OrangeRed}{#1}}
\usepackage{listings}
\lstset{basicstyle=\ttfamily}
\lstdefinestyle{tinyr}{language=R,basicstyle=\ttfamily\tiny,columns=fixed}
\lstdefinestyle{smallr}{language=R,basicstyle=\ttfamily\small,columns=fixed}
 \lstset{language=R,basicstyle=\scriptsize\ttfamily,columns=fixed}
\lstdefinelanguage{Maxima}{
  keywords={addrow,addcol,zeromatrix,ident,augcoefmatrix,ratsubst,diff,ev,tex,%
    with_stdout,nouns,express,depends,load,submatrix,div,grad,curl,%
    rootscontract,solve,part,assume,sqrt,integrate,abs,inf,exp},
  sensitive=true,
  comment=[n][\itshape]{/*}{*/}
}
\usepackage{pgfplots}
\pgfplotsset{compat=1.9}
\usepgflibrary{shapes}
\usetikzlibrary{automata}
\usetikzlibrary{arrows}
\usetikzlibrary{backgrounds}
\usetikzlibrary{fit}
\usetikzlibrary{shapes}
\pgfdeclarelayer{edgelayer}
\pgfdeclarelayer{nodelayer}
\pgfsetlayers{background,edgelayer,nodelayer,main}
\DeclareMathOperator\erf{erf}
\newcommand{\SE}{SE}
\newcommand{\SSE}{SSE}
\newcommand{\MSE}{MSE}
\newcommand{\SSR}{SSR}
\newcommand{\SST}{SST}
\newcommand{\SSxx}{SS_{xx}}
\newcommand{\SSxy}{SS_{xy}}
\newcommand{\SSyy}{SS_{yy}}
\newcommand{\VIF}{VIF}
\usepackage{xfrac} % for diagonal fractions in text

\newif\ifarticlemode
\articlemodetrue

\usepackage{booktabs} % better-looking tables
\usepackage{parskip}
\usepackage{ragged2e}
\newcommand{\source}[1]{%
  \nobreak\parbox[t]{0.9\linewidth}{\raggedleft #1}% Placing a quote source
}%
\usepackage{comment} % to process only what I'm working on now
\usepackage[group-separator={,}]{siunitx} % format large numerals
\newcommand{\stdBoxScale}{0.6}
%
% practice quizzes and exams
%
% declare a file for solutions
%\newwrite\tempfile
% use Alpha letters for answer choices to questions
\newcounter{choiceLetter}
\newcommand{\choice}{\item\addtocounter{enumi}{1}}
\newcommand{\CorrectChoice}{%
\Writetofile{solutions}{{\bfseries{\textcolor{solutionNcolor}{\arabic{question}.}}}\enspace
\textcolor{solutioncolor}{\theenumi\ is the correct answer.}}%
  \item\addtocounter{enumi}{1}%
}
\newenvironment{choices}{\begin{list}{\labelenumi}{\setlength\itemindent{0.0in}% 
                \setlength\listparindent{0.5in}%
                \setlength\labelwidth{0.50in}%
                \setlength\itemsep{0.0pt}%
                \setlength\parsep{0pt}%
                \setlength\leftmargin{0.5in}%
\setcounter{enumi}{1}
\renewcommand{\theenumi}{\Alph{enumi}}
\renewcommand{\labelenumi}{\theenumi.\ }
}}{\end{list}}
\newenvironment{oneparchoices}{\begin{list}{\labelenumi}{\setlength\itemindent{0.0in}% 
                \setlength\listparindent{0.5in}%
                \setlength\labelwidth{0.50in}%
                \setlength\itemsep{0.0pt}%
                \setlength\parsep{0pt}%
                \setlength\leftmargin{0.5in}%
\setcounter{enumi}{1}
\renewcommand{\theenumi}{\Alph{enumi}}
\renewcommand{\labelenumi}{\theenumi.\ }
}}{\end{list}}
% following works to color solutions and let them
% follow the questions
% \renewenvironment{solution}{%
% \par\setlength\parindent{-0.05in}%
% \begingroup\textcolor{solutioncolor}\bgroup%
% }{\egroup\endgroup}
\expandafter\let\csname solution\endcsname\relax
\expandafter\let\csname endsolution\endcsname\relax
\usepackage{answers}
\Newassociation{solution}{Solution}{solutions}
\renewenvironment{Solution}[1]{}{\par}
% terms list uses term, shortintertext, conc, shrt,lbl
\newcommand{\term}[1]{\noindent\textcolor{lbl}{#1}}
\usepackage{mathtools} % for \shortintertext
\definecolor{conc}{rgb}{0.10,0.5,0.10}
\definecolor{shrt}{rgb}{0.95,0.6,0.00}
\definecolor{lbl}{rgb}{0.00,0.5,0.60}
